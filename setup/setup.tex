\documentclass[a4paper]{article}

\usepackage[portuguese]{babel}
\usepackage[utf8]{inputenc}
\usepackage{graphicx,hyperref}
\usepackage{float}
\usepackage{proof,tikz}
\usepackage{amssymb,amsthm,stmaryrd}


\usepackage[edges]{forest}

%% ODER: format ==         = "\mathrel{==}"
%% ODER: format /=         = "\neq "
%
%
\makeatletter
\@ifundefined{lhs2tex.lhs2tex.sty.read}%
  {\@namedef{lhs2tex.lhs2tex.sty.read}{}%
   \newcommand\SkipToFmtEnd{}%
   \newcommand\EndFmtInput{}%
   \long\def\SkipToFmtEnd#1\EndFmtInput{}%
  }\SkipToFmtEnd

\newcommand\ReadOnlyOnce[1]{\@ifundefined{#1}{\@namedef{#1}{}}\SkipToFmtEnd}
\usepackage{amstext}
\usepackage{amssymb}
\usepackage{stmaryrd}
\DeclareFontFamily{OT1}{cmtex}{}
\DeclareFontShape{OT1}{cmtex}{m}{n}
  {<5><6><7><8>cmtex8
   <9>cmtex9
   <10><10.95><12><14.4><17.28><20.74><24.88>cmtex10}{}
\DeclareFontShape{OT1}{cmtex}{m}{it}
  {<-> ssub * cmtt/m/it}{}
\newcommand{\texfamily}{\fontfamily{cmtex}\selectfont}
\DeclareFontShape{OT1}{cmtt}{bx}{n}
  {<5><6><7><8>cmtt8
   <9>cmbtt9
   <10><10.95><12><14.4><17.28><20.74><24.88>cmbtt10}{}
\DeclareFontShape{OT1}{cmtex}{bx}{n}
  {<-> ssub * cmtt/bx/n}{}
\newcommand{\tex}[1]{\text{\texfamily#1}}	% NEU

\newcommand{\Sp}{\hskip.33334em\relax}


\newcommand{\Conid}[1]{\mathit{#1}}
\newcommand{\Varid}[1]{\mathit{#1}}
\newcommand{\anonymous}{\kern0.06em \vbox{\hrule\@width.5em}}
\newcommand{\plus}{\mathbin{+\!\!\!+}}
\newcommand{\bind}{\mathbin{>\!\!\!>\mkern-6.7mu=}}
\newcommand{\rbind}{\mathbin{=\mkern-6.7mu<\!\!\!<}}% suggested by Neil Mitchell
\newcommand{\sequ}{\mathbin{>\!\!\!>}}
\renewcommand{\leq}{\leqslant}
\renewcommand{\geq}{\geqslant}
\usepackage{polytable}

%mathindent has to be defined
\@ifundefined{mathindent}%
  {\newdimen\mathindent\mathindent\leftmargini}%
  {}%

\def\resethooks{%
  \global\let\SaveRestoreHook\empty
  \global\let\ColumnHook\empty}
\newcommand*{\savecolumns}[1][default]%
  {\g@addto@macro\SaveRestoreHook{\savecolumns[#1]}}
\newcommand*{\restorecolumns}[1][default]%
  {\g@addto@macro\SaveRestoreHook{\restorecolumns[#1]}}
\newcommand*{\aligncolumn}[2]%
  {\g@addto@macro\ColumnHook{\column{#1}{#2}}}

\resethooks

\newcommand{\onelinecommentchars}{\quad-{}- }
\newcommand{\commentbeginchars}{\enskip\{-}
\newcommand{\commentendchars}{-\}\enskip}

\newcommand{\visiblecomments}{%
  \let\onelinecomment=\onelinecommentchars
  \let\commentbegin=\commentbeginchars
  \let\commentend=\commentendchars}

\newcommand{\invisiblecomments}{%
  \let\onelinecomment=\empty
  \let\commentbegin=\empty
  \let\commentend=\empty}

\visiblecomments

\newlength{\blanklineskip}
\setlength{\blanklineskip}{0.66084ex}

\newcommand{\hsindent}[1]{\quad}% default is fixed indentation
\let\hspre\empty
\let\hspost\empty
\newcommand{\NB}{\textbf{NB}}
\newcommand{\Todo}[1]{$\langle$\textbf{To do:}~#1$\rangle$}

\EndFmtInput
\makeatother
%
%
%
%
%
%
% This package provides two environments suitable to take the place
% of hscode, called "plainhscode" and "arrayhscode". 
%
% The plain environment surrounds each code block by vertical space,
% and it uses \abovedisplayskip and \belowdisplayskip to get spacing
% similar to formulas. Note that if these dimensions are changed,
% the spacing around displayed math formulas changes as well.
% All code is indented using \leftskip.
%
% Changed 19.08.2004 to reflect changes in colorcode. Should work with
% CodeGroup.sty.
%
\ReadOnlyOnce{polycode.fmt}%
\makeatletter

\newcommand{\hsnewpar}[1]%
  {{\parskip=0pt\parindent=0pt\par\vskip #1\noindent}}

% can be used, for instance, to redefine the code size, by setting the
% command to \small or something alike
\newcommand{\hscodestyle}{}

% The command \sethscode can be used to switch the code formatting
% behaviour by mapping the hscode environment in the subst directive
% to a new LaTeX environment.

\newcommand{\sethscode}[1]%
  {\expandafter\let\expandafter\hscode\csname #1\endcsname
   \expandafter\let\expandafter\endhscode\csname end#1\endcsname}

% "compatibility" mode restores the non-polycode.fmt layout.

\newenvironment{compathscode}%
  {\par\noindent
   \advance\leftskip\mathindent
   \hscodestyle
   \let\\=\@normalcr
   \let\hspre\(\let\hspost\)%
   \pboxed}%
  {\endpboxed\)%
   \par\noindent
   \ignorespacesafterend}

\newcommand{\compaths}{\sethscode{compathscode}}

% "plain" mode is the proposed default.
% It should now work with \centering.
% This required some changes. The old version
% is still available for reference as oldplainhscode.

\newenvironment{plainhscode}%
  {\hsnewpar\abovedisplayskip
   \advance\leftskip\mathindent
   \hscodestyle
   \let\hspre\(\let\hspost\)%
   \pboxed}%
  {\endpboxed%
   \hsnewpar\belowdisplayskip
   \ignorespacesafterend}

\newenvironment{oldplainhscode}%
  {\hsnewpar\abovedisplayskip
   \advance\leftskip\mathindent
   \hscodestyle
   \let\\=\@normalcr
   \(\pboxed}%
  {\endpboxed\)%
   \hsnewpar\belowdisplayskip
   \ignorespacesafterend}

% Here, we make plainhscode the default environment.

\newcommand{\plainhs}{\sethscode{plainhscode}}
\newcommand{\oldplainhs}{\sethscode{oldplainhscode}}
\plainhs

% The arrayhscode is like plain, but makes use of polytable's
% parray environment which disallows page breaks in code blocks.

\newenvironment{arrayhscode}%
  {\hsnewpar\abovedisplayskip
   \advance\leftskip\mathindent
   \hscodestyle
   \let\\=\@normalcr
   \(\parray}%
  {\endparray\)%
   \hsnewpar\belowdisplayskip
   \ignorespacesafterend}

\newcommand{\arrayhs}{\sethscode{arrayhscode}}

% The mathhscode environment also makes use of polytable's parray 
% environment. It is supposed to be used only inside math mode 
% (I used it to typeset the type rules in my thesis).

\newenvironment{mathhscode}%
  {\parray}{\endparray}

\newcommand{\mathhs}{\sethscode{mathhscode}}

% texths is similar to mathhs, but works in text mode.

\newenvironment{texthscode}%
  {\(\parray}{\endparray\)}

\newcommand{\texths}{\sethscode{texthscode}}

% The framed environment places code in a framed box.

\def\codeframewidth{\arrayrulewidth}
\RequirePackage{calc}

\newenvironment{framedhscode}%
  {\parskip=\abovedisplayskip\par\noindent
   \hscodestyle
   \arrayrulewidth=\codeframewidth
   \tabular{@{}|p{\linewidth-2\arraycolsep-2\arrayrulewidth-2pt}|@{}}%
   \hline\framedhslinecorrect\\{-1.5ex}%
   \let\endoflinesave=\\
   \let\\=\@normalcr
   \(\pboxed}%
  {\endpboxed\)%
   \framedhslinecorrect\endoflinesave{.5ex}\hline
   \endtabular
   \parskip=\belowdisplayskip\par\noindent
   \ignorespacesafterend}

\newcommand{\framedhslinecorrect}[2]%
  {#1[#2]}

\newcommand{\framedhs}{\sethscode{framedhscode}}

% The inlinehscode environment is an experimental environment
% that can be used to typeset displayed code inline.

\newenvironment{inlinehscode}%
  {\(\def\column##1##2{}%
   \let\>\undefined\let\<\undefined\let\\\undefined
   \newcommand\>[1][]{}\newcommand\<[1][]{}\newcommand\\[1][]{}%
   \def\fromto##1##2##3{##3}%
   \def\nextline{}}{\) }%

\newcommand{\inlinehs}{\sethscode{inlinehscode}}

% The joincode environment is a separate environment that
% can be used to surround and thereby connect multiple code
% blocks.

\newenvironment{joincode}%
  {\let\orighscode=\hscode
   \let\origendhscode=\endhscode
   \def\endhscode{\def\hscode{\endgroup\def\@currenvir{hscode}\\}\begingroup}
   %\let\SaveRestoreHook=\empty
   %\let\ColumnHook=\empty
   %\let\resethooks=\empty
   \orighscode\def\hscode{\endgroup\def\@currenvir{hscode}}}%
  {\origendhscode
   \global\let\hscode=\orighscode
   \global\let\endhscode=\origendhscode}%

\makeatother
\EndFmtInput
%

\DeclareMathAlphabet{\mathkw}{OT1}{cmss}{bx}{n}


\newtheorem{Lemma}{Lemma}
\newtheorem{Theorem}{Theorem}
\theoremstyle{definition}
\newtheorem{Example}{Example}

\usepackage{xcolor}
\newcommand{\redFG}[1]{\textcolor[rgb]{0.6,0,0}{#1}}
\newcommand{\greenFG}[1]{\textcolor[rgb]{0,0.4,0}{#1}}
\newcommand{\blueFG}[1]{\textcolor[rgb]{0,0,0.8}{#1}}
\newcommand{\orangeFG}[1]{\textcolor[rgb]{0.8,0.4,0}{#1}}
\newcommand{\purpleFG}[1]{\textcolor[rgb]{0.4,0,0.4}{#1}}
\newcommand{\yellowFG}[1]{\textcolor{yellow}{#1}}
\newcommand{\brownFG}[1]{\textcolor[rgb]{0.5,0.2,0.2}{#1}}
\newcommand{\blackFG}[1]{\textcolor[rgb]{0,0,0}{#1}}
\newcommand{\whiteFG}[1]{\textcolor[rgb]{1,1,1}{#1}}
\newcommand{\yellowBG}[1]{\colorbox[rgb]{1,1,0.2}{#1}}
\newcommand{\brownBG}[1]{\colorbox[rgb]{1.0,0.7,0.4}{#1}}

\newcommand{\ColourStuff}{
  \newcommand{\red}{\redFG}
  \newcommand{\green}{\greenFG}
  \newcommand{\blue}{\blueFG}
  \newcommand{\orange}{\orangeFG}
  \newcommand{\purple}{\purpleFG}
  \newcommand{\yellow}{\yellowFG}
  \newcommand{\brown}{\brownFG}
  \newcommand{\black}{\blackFG}
  \newcommand{\white}{\whiteFG}
}

\newcommand{\MonochromeStuff}{
  \newcommand{\red}{\blackFG}
  \newcommand{\green}{\blackFG}
  \newcommand{\blue}{\blackFG}
  \newcommand{\orange}{\blackFG}
  \newcommand{\purple}{\blackFG}
  \newcommand{\yellow}{\blackFG}
  \newcommand{\brown}{\blackFG}
  \newcommand{\black}{\blackFG}
  \newcommand{\white}{\blackFG}
}

\ColourStuff

\newcommand{\D}[1]{\blue{\mathsf{#1}}}
\newcommand{\C}[1]{\red{\mathsf{#1}}}
\newcommand{\F}[1]{\green{\mathsf{#1}}}
\newcommand{\V}[1]{\black{\mathsf{#1}}}
\newcommand{\TC}[1]{\purple{\mathsf{#1}}}


\usepackage{fancyhdr}

\begin{document}

  \title{Configurando o ambiente para desenvolvimento Haskell}
  \author{Rodrigo Ribeiro}

  \maketitle

  \pagestyle{fancy}
  \fancyhf{}
  \lhead{Programa\c{c}\~ao Funcional}
  \rhead{Prof. Rodrigo Ribeiro}
  \rfoot{\thepage}
  \pagestyle{fancy}

  \section{Instalação}

  Para um bom desempenho na disciplina de Programação Funcional, é recomendado que
  você instale um ambiente para desenvolvimento Haskell. O que chamo de ``ambiente
  de desenvolvimento'' consiste de: 1) uma ferramenta para gerenciar bibliotecas,
  compiladores e projetos em Haskell e; 2) um editor de texto de sua preferência.

  Recomendo a instalação da ferramenta Haskell Stack, disponível gratuitamente em:
  
  \begin{center}
     \url{https://haskellstack.org}
  \end{center}

  \paragraph{Editores de texto} Existem vários editores de texto com suporte
  para Haskell. Eu recomendo:
  \begin{enumerate}
     \item Atom (com o pacote \texttt{language-haskell}). Disponível em:
     \begin{center}
         \url{https://atom.io}
     \end{center}
     \item Emacs: Esse editor é quase um sistema operacional...
           Caso deseje usar o Emacs, recomendo instalar o Haskell-mode. Como uso o emacs,
           fique a vontade para me perguntar sobre esse editor. O emacs pode ser baixado em:
           \begin{center}
             \url{https://www.gnu.org/software/emacs/}
           \end{center}
  \end{enumerate}

  \section{Meu primeiro projeto usando o Stack}

  No terminal de seu sistema operacional, execute o seguinte comando:

  \begin{tabbing}\tt
~~~stack~new~hello\char45{}world
\end{tabbing}

  que irá produzir a seguinte estrutura de arquivos:

\begin{forest}
  for tree={%
    folder,
    grow'=0,
    fit=band,
  }
  [.
    [LICENSE
    ]
    [Setup.hs
    ]
    [app
      [Main.hs]
    ]
    [src
      [Lib.hs]
    ]
    [stack.yaml]
    [test
      [Spec.hs]
    ]
  ]
\end{forest}

Após a criação desta estrutura de projeto, você pode executar o comando \text{\tt stack~setup}
que irá baixar o compilador Haskell (GHC), caso necessário.

\section{Hello World!}

Use seu editor de texto favorito e digite o seguinte programa:


\begin{hscode}\SaveRestoreHook
\column{B}{@{}>{\hspre}l<{\hspost}@{}}%
\column{E}{@{}>{\hspre}l<{\hspost}@{}}%
\>[B]{}\mathkw{module}\;\D{Main}\;\mathkw{where}{}\<[E]%
\\[\blanklineskip]%
\>[B]{}\F{main}\mathrel{=}\F{putStrLn}\;\orange{\mathsf{``Hello~world!\char34}}{}\<[E]%
\ColumnHook
\end{hscode}\resethooks

Após digitar o seguinte programa, sobreescreva o arquivo \text{\tt app\char47{}Main\char46{}hs} do projeto recém-criado.

Note que arquivos Haskell possuem a extensão \text{\tt \char46{}hs} ou \text{\tt \char46{}lhs}. Assim como em outras linguagens,
é importante que usemos o mesmo nome de arquivo para o módulo nele definido, caso contrário o compilador irá
apresentar uma mensagem de erro ao tentarmos importar esse módulo a partir de outro.

\subsection{Compilando e executando seu projeto}

O Haskell stack automatiza tarefas de gerenciamento de bibliotecas e compilação de programas. Por enquanto,
não vamos utilizar nenhuma biblioteca, além das previamente incluídas no compilador GHC. Para executar a
aplicação contida em um projeto, basta executar os seguintes comandos em ordem:

\begin{tabbing}\tt
~stack~build\\
\tt ~stack~exec~hello\char45{}world\char45{}exe
\end{tabbing}

O comando \text{\tt stack~build} irá compilar o programa atual e instalar bibliotecas por ele requeridas. Finalmente,
o comando \text{\tt stack~exec~hello\char45{}world\char45{}exe} irá executar o programa de nome \text{\tt hello\char45{}world\char45{}exe}. A utilidade destes
nomes é por permitir que existam mais de um executável em um mesmo projeto.

Além de gerenciar a compilação e execução de projetos, o stack permite usarmos o intepretador de Haskell, ghci, para
execução interativa do projeto. Para invocar o interpretador, basta usar o comando \text{\tt stack~ghci}, que irá apresentar
o seguinte \emph{prompt}:
\begin{tabbing}\tt
~\char42{}Main\char62{}
\end{tabbing}
Ao digitarmos o nome da função \ensuremath{\F{main}}, obtemos o resultado esperado.
\begin{tabbing}\tt
~Hello~world\char33{}
\end{tabbing}

\end{document}
